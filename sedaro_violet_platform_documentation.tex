\documentclass[12pt,a4paper]{article}
\usepackage[utf8]{inputenc}
\usepackage[T1]{fontenc}
\usepackage{geometry}
\usepackage{graphicx}
\usepackage{listings}
\usepackage{xcolor}
\usepackage{hyperref}
\usepackage{fancyhdr}
\usepackage{titlesec}
\usepackage{enumitem}
\usepackage{amsmath}
\usepackage{amssymb}

% Page setup
\geometry{margin=1in}
\pagestyle{fancy}
\fancyhf{}
\rhead{Sedaro Violet Platform}
\lhead{Technical Documentation}
\rfoot{\thepage}

% Code listing setup
\lstset{
    basicstyle=\ttfamily\footnotesize,
    breaklines=true,
    frame=single,
    numbers=left,
    numberstyle=\tiny,
    keywordstyle=\color{blue},
    commentstyle=\color{green!60!black},
    stringstyle=\color{red},
    backgroundcolor=\color{gray!10},
    showstringspaces=false,
    tabsize=2
}

% Title formatting
\titleformat{\section}{\Large\bfseries}{\thesection}{1em}{}
\titleformat{\subsection}{\large\bfseries}{\thesubsection}{1em}{}

\begin{document}

\title{\Huge\textbf{Sedaro Violet Platform}\\\large Technical Implementation Documentation}
\author{Development Team}
\date{\today}
\maketitle

\tableofcontents
\newpage

\section{Executive Summary}

The Sedaro Violet Platform is a Flask-based web application that provides an intuitive interface for managing and simulating agent templates using the Sedaro Python client. The platform features a spreadsheet-like interface for parameter management, real-time simulation monitoring, and comprehensive state persistence.

\section{Architecture Overview}

\subsection{Technology Stack}
\begin{itemize}
    \item \textbf{Backend}: Flask (Python web framework)
    \item \textbf{Frontend}: HTML5, CSS3, JavaScript (Vanilla)
    \item \textbf{API Client}: Sedaro Python SDK
    \item \textbf{State Management}: Browser localStorage
    \item \textbf{Styling}: Custom CSS with Violet Labs branding
\end{itemize}

\subsection{Core Components}
\begin{enumerate}
    \item Agent Template Management
    \item Interactive Property Editor
    \item BOM Spreadsheet Interface
    \item Simulation Control System
    \item State Persistence Layer
\end{enumerate}

\section{Key Features Implementation}

\subsection{1. Agent Template Integration}

\textbf{Purpose}: Direct integration with Sedaro agent templates for real-time property management.

\textbf{Code Location}: Lines 17-25, 138-154

\begin{lstlisting}[language=Python, caption=Agent Template Initialization]
# Define modules
sedaro = SedaroApiClient(api_key = API_KEY)
agent_template_branch = sedaro.agent_template(TEMP_AGENT_REPO_BRANCH_ID)

def get_agent_template_structure() -> Dict[str, Any]:
    """Get the complete structure of the agent template data"""
    try:
        agent_data = agent_template_branch.data
        return {
            'success': True,
            'data': agent_data,
            'keys': list(agent_data.keys()) if isinstance(agent_data, dict) else []
        }
    except Exception as e:
        return {
            'success': False,
            'error': str(e)
        }
\end{lstlisting}

\textbf{Key Features}:
\begin{itemize}
    \item Direct Sedaro client integration
    \item Template structure exploration
    \item Dynamic property discovery
    \item Real-time template refresh capability
\end{itemize}

\subsection{2. Interactive Property Editor}

\textbf{Purpose}: Visual interface for editing block and root-level properties with mutability detection.

\textbf{Code Location}: Lines 155-219, 250-292

\begin{lstlisting}[language=Python, caption=Property Update Functions]
def update_block_property(block_id: str, property_name: str, new_value: Any) -> Dict[str, Any]:
    """Update a specific property of a block using Sedaro client directly"""
    try:
        block = agent_template_branch.block(block_id)
        block.update(**{property_name: new_value})
        updated_value = getattr(block, property_name, new_value)
        
        return {
            'success': True,
            'message': f'Successfully updated {property_name} to {updated_value}',
            'updated_value': updated_value
        }
    except Exception as e:
        return {
            'success': False,
            'error': str(e)
        }

def get_property_mutability(block_id: Optional[str] = None, property_name: Optional[str] = None) -> Dict[str, Any]:
    """Check if a property is mutable by attempting to get its metadata"""
    try:
        if block_id:
            block = agent_template_branch.block(block_id)
            block_data = block.data
            
            if property_name not in block_data:
                return {
                    'success': False,
                    'error': f'Property {property_name} not found in block {block_id}'
                }
            
            immutable_properties = ['id', 'type', '_id', '_type', '_version', '_created', '_updated']
            is_mutable = property_name not in immutable_properties
            
            return {
                'success': True,
                'mutable': is_mutable,
                'reason': 'Immutable system property' if not is_mutable else 'Editable property'
            }
    except Exception as e:
        return {
            'success': False,
            'error': str(e)
        }
\end{lstlisting}

\textbf{Key Features}:
\begin{itemize}
    \item Direct Sedaro client property updates
    \item Mutability detection for UI feedback
    \item Type conversion handling
    \item Comprehensive error handling
\end{itemize}

\subsection{3. BOM Spreadsheet Interface}

\textbf{Purpose}: Spreadsheet-like interface for managing input variables and associating them with template properties.

\textbf{Code Location}: Lines 2074-2150, 2400-2500

\begin{lstlisting}[language=JavaScript, caption=Spreadsheet Row Management]
function addSpreadsheetRow(restoreRowId = null, restoreData = null) {
    const tbody = document.getElementById('bomTableBody');
    const rowId = restoreRowId || Math.floor(Math.random() * 1000000);
    
    const row = document.createElement('tr');
    row.id = `row-${rowId}`;
    row.innerHTML = `
        <td>
            <input type="text" class="spreadsheet-input" placeholder="Variable name" 
                   onchange="updateSpreadsheetData('${rowId}', 'name', this.value)">
        </td>
        <td>
            <input type="text" class="spreadsheet-input" placeholder="Input value" 
                   onchange="updateSpreadsheetData('${rowId}', 'value', this.value)">
        </td>
        <td>
            <div class="property-selector">
                <input type="text" placeholder="Click to select template property" readonly
                       onclick="showPropertyDropdown('${rowId}')">
                <div class="property-dropdown" id="dropdown-${rowId}"></div>
            </div>
        </td>
        <td>
            <div class="action-buttons">
                <button class="action-btn save" onclick="saveProperty('${rowId}')">
                    💾 Save
                </button>
                <button class="action-btn delete" onclick="deleteRow('${rowId}')">
                    🗑️
                </button>
            </div>
        </td>
    `;
    
    tbody.appendChild(row);
    
    if (restoreData) {
        spreadsheetData[rowId] = restoreData;
        // Restore input values
        const nameInput = row.querySelector('td:nth-child(1) input');
        const valueInput = row.querySelector('td:nth-child(2) input');
        const propertyInput = row.querySelector('td:nth-child(3) input');
        
        if (nameInput && restoreData.name) nameInput.value = restoreData.name;
        if (valueInput && restoreData.value) valueInput.value = restoreData.value;
        if (propertyInput && restoreData.associatedProperty) {
            propertyInput.value = restoreData.associatedProperty.displayName;
        }
    } else {
        spreadsheetData[rowId] = {
            name: '',
            value: '',
            associatedProperty: null,
            lastAppliedValue: null
        };
    }
}
\end{lstlisting}

\textbf{Key Features}:
\begin{itemize}
    \item Dynamic row addition/deletion
    \item Property association via dropdown
    \item Auto-trigger simulation capability
    \item State restoration support
\end{itemize}

\subsection{4. Auto-Trigger Simulation System}

\textbf{Purpose}: Automatically trigger simulations when variables change, with comprehensive error handling.

\textbf{Code Location}: Lines 2400-2500

\begin{lstlisting}[language=JavaScript, caption=Auto-Trigger Logic]
async function applySpreadsheetChanges() {
    const autoSimEnabled = document.getElementById('autoSimToggle').checked;
    if (!autoSimEnabled) return;
    
    let hasChanges = false;
    let hasErrors = false;
    
    for (const [rowId, data] of Object.entries(spreadsheetData)) {
        if (data.name && data.value && data.associatedProperty) {
            const currentValue = parseValue(data.value);
            const lastValue = data.lastAppliedValue;
            
            // Check if value has actually changed locally
            if (JSON.stringify(currentValue) !== JSON.stringify(lastValue)) {
                try {
                    let updateResponse;
                    
                    if (data.associatedProperty.blockId) {
                        // Update block property
                        updateResponse = await fetch('/api/update_property', {
                            method: 'POST',
                            headers: { 'Content-Type': 'application/json' },
                            body: JSON.stringify({
                                block_id: data.associatedProperty.blockId,
                                property_name: data.associatedProperty.name,
                                new_value: currentValue
                            })
                        });
                    } else {
                        // Update root property
                        updateResponse = await fetch('/api/update_root_property', {
                            method: 'POST',
                            headers: { 'Content-Type': 'application/json' },
                            body: JSON.stringify({
                                property_name: data.associatedProperty.name,
                                new_value: currentValue
                            })
                        });
                    }
                    
                    const updateResult = await updateResponse.json();
                    
                    if (updateResult.success) {
                        const actualChanged = updateResult.updated_value !== data.lastAppliedValue;
                        
                        if (actualChanged) {
                            hasChanges = true;
                            showToast(`Updated ${data.name} to ${data.value}`, 'success');
                        } else {
                            showToast(`No change needed for ${data.name}`, 'info');
                        }
                        
                        spreadsheetData[rowId].lastAppliedValue = updateResult.updated_value;
                    } else {
                        hasErrors = true;
                        showToast(`Failed to update ${data.name}: ${updateResult.error}`, 'error');
                    }
                } catch (error) {
                    hasErrors = true;
                    showToast(`Failed to update ${data.name}: ${error.message}`, 'error');
                }
            }
        }
    }
    
    // Only trigger simulation if we actually had changes AND no errors occurred
    if (hasChanges && !hasErrors) {
        setTimeout(() => {
            startSimulation();
        }, 1000);
    } else if (hasErrors) {
        showToast('Simulation skipped due to update errors', 'error');
    }
}
\end{lstlisting}

\textbf{Key Features}:
\begin{itemize}
    \item Change detection and validation
    \item Comprehensive error handling
    \item Conditional simulation triggering
    \item User feedback through toast notifications
\end{itemize}

\subsection{5. Real-Time Simulation Monitoring}

\textbf{Purpose}: Non-blocking simulation progress monitoring with automatic polling and timeout protection.

\textbf{Code Location}: Lines 1325-1400

\begin{lstlisting}[language=JavaScript, caption=Progress Monitoring System]
function startProgressMonitoring(simulationId) {
    // Clear any existing interval
    if (progressInterval) {
        clearInterval(progressInterval);
    }
    
    // Update immediately
    updateProgress(simulationId);
    
    // Then update every 2 seconds
    progressInterval = setInterval(async () => {
        const isComplete = await updateProgress(simulationId);
        if (isComplete) {
            clearInterval(progressInterval);
            progressInterval = null;
        }
    }, 2000);
    
    // Safety timeout: stop polling after 30 minutes
    setTimeout(() => {
        if (progressInterval) {
            console.warn('Stopping simulation monitoring due to timeout (30 minutes)');
            clearInterval(progressInterval);
            progressInterval = null;
            showToast('Simulation monitoring stopped due to timeout', 'info');
            
            const statusText = document.getElementById('statusText');
            if (statusText) {
                statusText.textContent = '⏰ Simulation monitoring timed out after 30 minutes';
            }
        }
    }, 30 * 60 * 1000); // 30 minutes
}

async function updateProgress(simulationId) {
    try {
        const response = await fetch(`/api/simulation_status?simulation_id=${simulationId}`);
        const result = await response.json();
        
        if (result.success) {
            const progressFill = document.getElementById('progressFill');
            const progressText = document.getElementById('progressText');
            const statusText = document.getElementById('statusText');
            
            // Update progress bar
            const progress = result.progress || 0;
            progressFill.style.width = `${progress}%`;
            progressText.textContent = `${Math.round(progress)}%`;
            
            // Update status text
            statusText.textContent = result.message;
            
            // Check if simulation is complete
            if (result.is_complete) {
                if (result.status === 'SUCCEEDED') {
                    statusText.textContent = '✅ Simulation completed successfully!';
                    showToast('Simulation completed successfully!', 'success');
                } else {
                    statusText.textContent = `❌ Simulation ${result.status.toLowerCase()}`;
                    showToast(`Simulation ${result.status.toLowerCase()}`, 'error');
                }
                
                // Hide progress bar and abort button, but keep status message
                const progressBar = document.querySelector('.progress-container');
                const abortBtn = document.getElementById('abortBtn');
                if (progressBar) progressBar.style.display = 'none';
                if (abortBtn) abortBtn.style.display = 'none';
                
                // Reset simulation ID to allow new simulations
                currentSimulationId = null;
                
                return true; // Stop polling
            }
            
            return false; // Continue polling
        } else {
            console.error('Failed to get simulation status:', result.error);
            showToast('Failed to get simulation status, stopping monitoring', 'error');
            return true; // Stop polling
        }
    } catch (error) {
        console.error('Error checking simulation status:', error);
        showToast('Error checking simulation status, stopping monitoring', 'error');
        return true; // Stop polling
    }
}
\end{lstlisting}

\textbf{Key Features}:
\begin{itemize}
    \item Non-blocking progress updates
    \item Automatic polling with 2-second intervals
    \item 30-minute safety timeout
    \item Comprehensive error handling
    \item UI state management
\end{itemize}

\subsection{6. State Persistence System}

\textbf{Purpose}: Prevent data loss on page refresh by saving and restoring application state.

\textbf{Code Location}: Lines 1170-1250

\begin{lstlisting}[language=JavaScript, caption=State Persistence Implementation]
function saveState() {
    const state = {
        currentSimulationId: currentSimulationId,
        spreadsheetData: spreadsheetData,
        autoSimEnabled: document.getElementById('autoSimToggle')?.checked || false
    };
    localStorage.setItem('sedaroVioletState', JSON.stringify(state));
}

function loadState() {
    try {
        const savedState = localStorage.getItem('sedaroVioletState');
        if (savedState) {
            const state = JSON.parse(savedState);
            
            // Restore simulation ID
            if (state.currentSimulationId) {
                currentSimulationId = state.currentSimulationId;
                // Check if simulation is still running
                checkAndRestoreSimulation(state.currentSimulationId);
            }
            
            // Restore spreadsheet data
            if (state.spreadsheetData) {
                spreadsheetData = state.spreadsheetData;
                restoreSpreadsheetUI();
            }
            
            // Restore auto-sim toggle
            if (state.autoSimEnabled !== undefined) {
                const autoSimToggle = document.getElementById('autoSimToggle');
                if (autoSimToggle) {
                    autoSimToggle.checked = state.autoSimEnabled;
                }
            }
            
            console.log('State restored from localStorage');
        }
    } catch (error) {
        console.error('Error loading state:', error);
        // Clear corrupted state
        localStorage.removeItem('sedaroVioletState');
    }
}

async function checkAndRestoreSimulation(simulationId) {
    try {
        const response = await fetch(`/api/simulation_status?simulation_id=${simulationId}`);
        const result = await response.json();
        
        if (result.success && !result.is_complete) {
            // Simulation is still running, restore progress monitoring
            const progressDiv = document.getElementById('simulationProgress');
            if (progressDiv) {
                progressDiv.style.display = 'block';
            }
            startProgressMonitoring(simulationId);
            showToast('Restored simulation monitoring', 'info');
        } else {
            // Simulation is complete or doesn't exist, clear the ID
            currentSimulationId = null;
            saveState();
        }
    } catch (error) {
        console.error('Error checking simulation status:', error);
        currentSimulationId = null;
        saveState();
    }
}

function setupStatePersistence() {
    // Save state every 5 seconds
    setInterval(saveState, 5000);
    
    // Save state on page unload
    window.addEventListener('beforeunload', saveState);
    
    // Save state when spreadsheet data changes
    const originalUpdateSpreadsheetData = updateSpreadsheetData;
    updateSpreadsheetData = function(rowId, field, value) {
        originalUpdateSpreadsheetData(rowId, field, value);
        saveState();
    };
}
\end{lstlisting}

\textbf{Key Features}:
\begin{itemize}
    \item Automatic state saving every 5 seconds
    \item State restoration on page load
    \item Simulation state recovery
    \item Spreadsheet data persistence
    \item Error handling for corrupted state
\end{itemize}

\section{API Endpoints}

\subsection{Core Endpoints}

\begin{enumerate}
    \item \texttt{/api/blocks} - Get all blocks and properties
    \item \texttt{/api/update\_property} - Update block properties
    \item \texttt{/api/update\_root\_property} - Update root-level properties
    \item \texttt{/api/simulate} - Start simulation
    \item \texttt{/api/simulation\_status} - Get simulation progress
    \item \texttt{/api/abort\_simulation} - Abort running simulation
    \item \texttt{/api/template\_structure} - Get template structure
    \item \texttt{/api/refresh\_template} - Refresh template connection
\end{enumerate}

\section{User Interface Features}

\subsection{Visual Design}
\begin{itemize}
    \item \textbf{Branding}: Violet Labs color scheme (\#c3adff primary)
    \item \textbf{Responsive Layout}: Mobile-friendly design
    \item \textbf{Toast Notifications}: Real-time user feedback
    \item \textbf{Progress Indicators}: Visual simulation progress
    \item \textbf{Interactive Elements}: Hover effects and transitions
\end{itemize}

\subsection{User Experience}
\begin{itemize}
    \item \textbf{Intuitive Workflow}: Step-by-step parameter management
    \item \textbf{Error Prevention}: Validation and confirmation dialogs
    \item \textbf{State Recovery}: Seamless refresh experience
    \item \textbf{Real-time Updates}: Live progress monitoring
\end{itemize}

\section{Error Handling Strategy}

\subsection{Comprehensive Error Management}
\begin{enumerate}
    \item \textbf{API Error Handling}: Graceful degradation for network issues
    \item \textbf{Validation Errors}: User-friendly error messages
    \item \textbf{State Corruption}: Automatic recovery mechanisms
    \item \textbf{Simulation Failures}: Detailed error reporting
\end{enumerate}

\section{Performance Optimizations}

\subsection{Efficiency Measures}
\begin{itemize}
    \item \textbf{Lazy Loading}: Template properties loaded on demand
    \item \textbf{Debounced Updates}: Reduced API calls during rapid changes
    \item \textbf{Memory Management}: Proper cleanup of intervals and event listeners
    \item \textbf{Caching}: Local storage for state persistence
\end{itemize}

\section{Development Best Practices}

\subsection{Code Quality}
\begin{itemize}
    \item \textbf{Modular Design}: Separated concerns and reusable components
    \item \textbf{Error Boundaries}: Comprehensive try-catch blocks
    \item \textbf{Documentation}: Inline comments and function documentation
    \item \textbf{Type Safety}: Proper data type handling
\end{itemize}

\section{Future Enhancements}

\subsection{Planned Features}
\begin{enumerate}
    \item \textbf{Batch Operations}: Multi-property updates
    \item \textbf{Template Versioning}: Historical data tracking
    \item \textbf{Advanced Analytics}: Simulation result analysis
    \item \textbf{User Authentication}: Multi-user support
    \item \textbf{API Rate Limiting}: Enhanced performance monitoring
\end{enumerate}

\section{Conclusion}

The Sedaro Violet Platform represents a comprehensive solution for agent template management and simulation control. The implementation demonstrates modern web development practices with a focus on user experience, reliability, and maintainability.

Key achievements include:
\begin{itemize}
    \item Seamless Sedaro API integration
    \item Intuitive spreadsheet interface
    \item Robust state persistence
    \item Real-time simulation monitoring
    \item Comprehensive error handling
\end{itemize}

The platform successfully bridges the gap between complex simulation parameters and user-friendly interface design, enabling efficient workflow management for Sedaro agent templates.

\end{document} 